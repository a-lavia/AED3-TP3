\section{Introducción}
% GENERAL: Describir detalladamente el problema a resolver dando ejemplos del mismo y sus soluciones.
% 1 el enunciado informal del problema (si existe), brevemente.
% 2 el enunciado formal del problema.
% 3 ejemplos, casos interesantes, explicaciones en palabras  que ayuden a comprender qu ́e problema se est ́a tratando de resolver.

En este problema, Brian desea convertirse en maestro Pokémon. Se conoce como pokemones a unas criaturas que habitan el mundo, las personas pueden capturarlos y tenerlos como mascotas o entrenarlos para pelear contra otros, tanto de entrenados como salvajes. Para convertirse en maestro Pokémon, se debe vencer a los entrenadores que controlan cada gimnasio del mundo, donde un gimnasio es simplemente un lugar donde hay un entrenador y un pok\'emon para vencer.

Vencer un gimnasio no es fácil debido a que nuestros pokemones se debilitan en las batallas; para paliar esto, existen pociones que curan a nuestros pokemones y les permiten continuar batallando. \'Estas se pueden conseguir en estaciones conocidas como pokeparadas, en las que entregan exactamente tres. Brian cuenta con una mochila en la que puede cargar hasta cierta cantidad, por lo que si pasa por una pokeparada y las tres no entran en su mochila deberá descartar las sobrantes. Para saber si podemos vencer un gimnasio, contamos de antemano con el número de pociones que son requeridas para lograrlo. No es posible visitar un gimnasio si no podemos vencerlo.

El objetivo es encontrarar la manera de vencer todos los gimnasios sin pasar por un gimnasio o una pokeparada más de una vez, recorriendo la mínima distancia.

Podemos modelar el problema con un grafo donde cada gimnasio y cada pokeparada se corresponde con un nodo. Es un grafo completo ya que podemos ir desde cualquier posici\'on del plano a cualquier otra, y el peso de las aristas se corresponde con la distancia euclideana\footnote{\url{https://es.wikipedia.org/wiki/Distancia_euclidiana}} calculada utilizando las coordenadas de los nodos. Luego, nuestro objetivo es hallar un camino que pase por todos los nodos que representan a los gimnasios y todas las pokeparadas necesarias para vencerlos, minimizando la distancia del recorrido.

\begin{comment}
Para pasar por un gimnasio debemos poder vencerlo, y para esto necesitaremos tener la cantidad de pociones que el gimnasio requiere. Para tener estas pociones debemos pasar por poke paradas y guardar las pociones que obtenemos en nuestra mochila, descartando las que sobren en caso de que la mochila se llene.
\end{comment}
\begin{comment}
Formalmente, esto es equivalente a: dado dos conjuntos de tuplas en un plano y un número natural, donde uno de los conjuntos es una tripla de tres enteros que representa a los gimnasios donde las primeras dos posiciones de la misma representan las coordenadas del plano y la última representa la cantidad de pociones necesarias para llegar al nodo; el otro conjunto es una tupla de dos enteros que representa a las poke paradas donde los dos enteros representan la coordenada de la misma en el plano; y el natural $q = 0$ que representa a la mochila, donde éste está acotado por otro natural $k$ que representa el límite de peso de la misma, se quiere encontrar un camino que cubra todos los puntos del conjunto que representa a los gimnasios con distancia mínima entre ellos.
\end{comment}

%   \subsection{Caso de ejemplo}

% Un caso simple en un eje de coordenadas
% Necesito el Backtracking para buscar una solucion optima y mostrarla

\begin{comment}
\subsection{Modelo del problema}

Para resolver este problema lo modelamos en forma de grafo completo donde cada nodo es un gimnasio o una poke parada y cada arista tiene como peso la distancia que hay entre dos nodos. Si el nodo es un gimnasio también posée un atributo que es la cantidad de pociones que se necesita para ser vencido y si es una poke parada no posée nada ya que por cada una se reciben tres pociones.

Entonces el caso de ejemplo quedaría de la siguiente forma:
\end{comment}

% Mismo caso pero modelado como si fuera un grafo